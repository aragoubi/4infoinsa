% Définit si on compile en recto (pour la lecture sur écran) ou en rectoverso (pour l'impression)
\providecommand{\VarRectoVerso}{oneside}

\documentclass[french,11pt]{article}

\usepackage[T1]{fontenc}
\usepackage[utf8]{inputenc}
\usepackage{babel}
\usepackage{fullpage}
\usepackage{lmodern}
\usepackage{graphicx}
\usepackage{enumerate}
\usepackage{xspace}
\usepackage{amsmath}
\usepackage{amsfonts}
%\usepackage{mathenv}
\usepackage{amsthm}

\newtheorem{remarque}{Remarque}[subsubsection]
\newtheorem{definition}{Définition}[subsubsection]
\newtheorem{proposition}{Proposition}[subsubsection]
\newtheorem{theoreme}{Théorème}[subsubsection]
\newtheorem{exemple}{Exemple}[subsubsection]

%\usepackage{verbatim}
%\usepackage{moreverb} %pour utiliser plus de fonctions de verbatim

\newcommand{\chapterEt}[1]{\chapter*{#1}\addcontentsline{toc}{chapter}{#1}}
\newcommand{\sectionEt}[1]{\section*{#1}\addcontentsline{toc}{section}{#1}}

\newcommand{\R}{\mathbb{R}}
\newcommand{\x}{\bar{x}}



%\input{./naming}

\title{Compléxité}
\author{Danielle \textsc{Quichaud}}

%%%%%%%%%%%%%%%%%%%%%%%%%%%%%%%%%

\begin{document}

\maketitle

\tableofcontents

\section{TD1}

\begin{exercice}{Fibonacci}

\[ \begin{cases} u_n = u_{n-1} + u_{n-2} \\ u_1 = 1 \\ u_0= 0 \end{cases} \]

\textit{Équation caractéristique} 
\[x^2-x-1 = 0\]
\[ \Delta = 1+4=5 >0 \]
\[ \textit{2 racines distincts } \frac{1 \pm \sqrt{5}}{2} \]
\[ u_n = a \left(\frac{1 - \sqrt{5}}{2}\right)^n + b \left(\frac{1 + \sqrt{5}}{2}\right)^n \]

\textit{Calcul de a et b} 
\[ u_0 = 0 = a +b \]
\[ u_1 = 1 = a \left(\frac{1 - \sqrt{5}}{2}\right) + b \left(\frac{1 + \sqrt{5}}{2}\right) \]
\[ \Rightarrow \]
\[ a + b = 0 \Leftrightarrow b = -a \]
\[ \sqrt(5) (b-a) = 2 \]
\[ \Leftrightarrow \]
\[ b = \frac{1}{\sqrt{5}} \]
\[ a = -\frac{1}{\sqrt{5}} \]

\[ u_n = \frac{1}{\sqrt{5}} \left[ \left( \frac{1 + \sqrt{5}}{2} \right)^n - \left( \frac{1 - \sqrt{5}}{2} \right)^n \right] \]


\end{exercice}

\begin{exercice}
\[ \begin{cases} u_n = 4 u_{n-1} - 4 u_{n-2} + 2 n \\ u_1 = 1 \\ u_0 = 0 \end{cases} \]
On a bien $f(n)$ de la forme $b^n P(n)$ avec $b = 1$ et $P(n) = 2n$, $d^o P = 1$.

\textit{Équation caractéristique} 
\[ (x^2-4x+4)(x-1)^2 = 0 \]
\[ \Ra (x-2)^2(x-1)^2 = 0 \]
\[ \Ra \textit{ 2 et 1 racines doubles} \]
\[ \Ra u_n=(an+b)2^n+(cn+d) \]
\[ \textit{(sol gen + sol part)} \]

\textit{Calcul de c et d} en écrivant que $u_n=cn+d$ est une solution particulière de $t$.
\[ cn+d=4[c(n-1)+d]-4[c(n-2)+d]+2n \] 
\[ cn+d+4c-4d-8c+4d=2n \]
\[ cn+d-4c=2n \]
\[ \Ra \begin{cases} c=2 \\ d-4c=0 \Ra d=8 \end{cases} \]
\[ \Ra u_n=(an+b)2^n+2n+8 \]

\textit{Calcul de a et b}
\[ \begin{cases} u_0=0=b+8 \Ra b=-8 \\u_1=1=2(a+b)+10 \Ra 2a=7 \Ra a=\frac{7}{2} \end{cases} \]

\[ \Ra u_n=(\frac{7}{2}n-2)2^n+2n+8 \]

\end{exercice}

\begin{exercice}
\[ \begin{cases} u_n=u_{n-1}+u_{n-3}-u_{n-4} \textit{ pour } n\ge 4 \\ u_n=n \textit{ pour } 0 \le n \le 5 \end{cases} \]

\textit{Équation caractéristique} 
\[ x^4-x^3-x+1=0 \]
\[ \textit{solution évidente : } 1 \]
\[ (x-1)(x^3-1)=0 \]
\[ (x-1)(x-1)(x^2+x+1)=0 \]
\[ (x-1)^2(x^2+x+1)=0 \]
\[ x^2+x+1 \textit{ n'a pas de solution réelles } \]
\[ 2 \textit{ racines complexes } \frac{-1 \pm 
i \sqrt{3}}{2} = \pm e^{i\frac{2\pi}{3}} \]
\[ \textit{ce sont les racines cubiques de l'unité : } j \textit{ et } j^2 \]

\[ u_n=(an+b)+cj^n+dj^{2n} \]

Calcul de a, b, c et d à partir des conditions initiales
\[ \begin{cases} u_0=0=b+c+d \\ 
u_1=1=a+b+cj+dj^2 \\
u_2=2=2a+b+cj^2+dj \\
u_3=3=3a+b+c+d \end{cases} \]

\[ \Ra \begin{cases} a=1 \\ 
b+c+d=0 \\
b+cj+dj^2=0 \\
b+cj^2+dj=0 \end{cases} \]

\[ \Ra \begin{cases} a=1 \\ 
b=-c-d \\
c(j-1)+d(j^2-1)=0 \\
c(j^2-1)+d(j-1)=0 \end{cases} \]

\[ \Ra \begin{cases} a=1 \\ 
b=c=d=0 \end{cases} \]
 
\end{exercice}
 
\begin{exercice}

\[ \begin{cases} u_n=3u_{n-1}^2 \textit{ pour } n \ge 1 \\ u_0=1 \end{cases} \]

Il faut se ramener à une équation de récurrence linéaire qu'on résoud par la méthode des l'équation caractéristique. On passe donc par les logarithmes.

\[ \log u_n =n\log 3+2\log u_{n-1} \]

On change de varible : $v_n=\log u_n$.
\[ \begin{cases} v_n=2v_{n-1}+n\log 3 \\
v_0=\log u_0=0 \end{cases} \]

Le second membre est bien de la forme $b^nP(n)$ avec $b=1$ et $P(n)=n\log 3$ ($d^o P=1$), on utilise donc la méthode de l'équation caractéristique.

\[ (x-2)(x-2)^2=0 \]
\[ v_n=a2^n+bn+c \]

Calcul de b et de c
\[ bn+c=2[b(n-1)+c]+n\log 3 \]
\[ -bn-c-2b=n\log 3\]
\[ \begin{cases} b=-\log 3 \\ c+2b=0 \Ra c=-2b=2\log 3 \end{cases} \]

Calcul de a à partir de $v_0$
\[ v_0=0=a+c \Ra a=-2\log 3 \]
\[ v_n=-2^{n+1}\log 3-n\log 3+2\log3 \]
\[ v_n=[-2^{n+1}-n+2]\log 3 \]
\[ \Ra u_n=e^{v_n}=e^{[-2^{n+1}-n+2]\log 3} \]
\[ u_n=3^{-2^{n+1}-n+2} \]

\end{exercice}

\begin{exercice}

\[ \begin{cases} T(n)=aT(\frac{n}{n})+g(n) \\ T(1)=1 \end{cases} \]

On obtient ce genre d'équations avec des algos du type diviser pour régner. Problème décomposé en sous problèmes de taille $\frac{n}{b}$. $g(n)$ est le cout de la décomposition, et de la recomposition.

Hypothèse : $n$ puissance de $b$ : $n=b$
\[ \begin{cases} T(b^k)=aT(b^{k-1})+g(b^k) \\ T(b^0)=1 \end{cases} \]

On fait ler changement de variable : $u_k=T(b^k)$
%TODO


\end{exercice}

\end{document}
