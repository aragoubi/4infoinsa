\providecommand{\VarRectoVerso}{\oneside}
\documentclass[french,11pt]{article}

\usepackage[T1]{fontenc}
\usepackage[utf8]{inputenc}
\usepackage{babel}
\usepackage{fullpage}
\usepackage{lmodern}
\usepackage{graphicx}
\usepackage{enumerate}
\usepackage{xspace}
\usepackage{amsmath}
\usepackage{amsfonts}
%\usepackage{mathenv}
\usepackage{amsthm}

\newtheorem{remarque}{Remarque}[subsubsection]
\newtheorem{definition}{Définition}[subsubsection]
\newtheorem{proposition}{Proposition}[subsubsection]
\newtheorem{theoreme}{Théorème}[subsubsection]
\newtheorem{exemple}{Exemple}[subsubsection]

%\usepackage{verbatim}
%\usepackage{moreverb} %pour utiliser plus de fonctions de verbatim

\newcommand{\chapterEt}[1]{\chapter*{#1}\addcontentsline{toc}{chapter}{#1}}
\newcommand{\sectionEt}[1]{\section*{#1}\addcontentsline{toc}{section}{#1}}

\newcommand{\R}{\mathbb{R}}
\newcommand{\x}{\bar{x}}



\title{Complexité}
\author{Danielle Quichaud}
\usepackage{amsmath}
\begin{document}
\maketitle
\tableofcontents
\section{TD1}

\begin{exercice}{Fibonacci}

\[ \begin{cases} u_n = u_{n-1} + u_{n-2} \\ u_1 = 1 \\ u_0= 0 \end{cases} \]

\textit{Équation caractéristique} 
\[x^2-x-1 = 0\]
\[ \Delta = 1+4=5 >0 \]
\[ \textit{2 racines distincts } \frac{1 \pm \sqrt{5}}{2} \]
\[ u_n = a \left(\frac{1 - \sqrt{5}}{2}\right)^n + b \left(\frac{1 + \sqrt{5}}{2}\right)^n \]

\textit{Calcul de a et b} 
\[ u_0 = 0 = a +b \]
\[ u_1 = 1 = a \left(\frac{1 - \sqrt{5}}{2}\right) + b \left(\frac{1 + \sqrt{5}}{2}\right) \]
\[ \Rightarrow \]
\[ a + b = 0 \Leftrightarrow b = -a \]
\[ \sqrt(5) (b-a) = 2 \]
\[ \Leftrightarrow \]
\[ b = \frac{1}{\sqrt{5}} \]
\[ a = -\frac{1}{\sqrt{5}} \]

\[ u_n = \frac{1}{\sqrt{5}} \left[ \left( \frac{1 + \sqrt{5}}{2} \right)^n - \left( \frac{1 - \sqrt{5}}{2} \right)^n \right] \]


\end{exercice}

\begin{exercice}
\[ \begin{cases} u_n = 4 u_{n-1} - 4 u_{n-2} + 2 n \\ u_1 = 1 \\ u_0 = 0 \end{cases} \]
On a bien $f(n)$ de la forme $b^n P(n)$ avec $b = 1$ et $P(n) = 2n$, $d^o P = 1$.

\textit{Équation caractéristique} 
\[ (x^2-4x+4)(x-1)^2 = 0 \]
\[ \Ra (x-2)^2(x-1)^2 = 0 \]
\[ \Ra \textit{ 2 et 1 racines doubles} \]
\[ \Ra u_n=(an+b)2^n+(cn+d) \]
\[ \textit{(sol gen + sol part)} \]

\textit{Calcul de c et d} en écrivant que $u_n=cn+d$ est une solution particulière de $t$.
\[ cn+d=4[c(n-1)+d]-4[c(n-2)+d]+2n \] 
\[ cn+d+4c-4d-8c+4d=2n \]
\[ cn+d-4c=2n \]
\[ \Ra \begin{cases} c=2 \\ d-4c=0 \Ra d=8 \end{cases} \]
\[ \Ra u_n=(an+b)2^n+2n+8 \]

\textit{Calcul de a et b}
\[ \begin{cases} u_0=0=b+8 \Ra b=-8 \\u_1=1=2(a+b)+10 \Ra 2a=7 \Ra a=\frac{7}{2} \end{cases} \]

\[ \Ra u_n=(\frac{7}{2}n-2)2^n+2n+8 \]

\end{exercice}

\begin{exercice}
\[ \begin{cases} u_n=u_{n-1}+u_{n-3}-u_{n-4} \textit{ pour } n\ge 4 \\ u_n=n \textit{ pour } 0 \le n \le 5 \end{cases} \]

\textit{Équation caractéristique} 
\[ x^4-x^3-x+1=0 \]
\[ \textit{solution évidente : } 1 \]
\[ (x-1)(x^3-1)=0 \]
\[ (x-1)(x-1)(x^2+x+1)=0 \]
\[ (x-1)^2(x^2+x+1)=0 \]
\[ x^2+x+1 \textit{ n'a pas de solution réelles } \]
\[ 2 \textit{ racines complexes } \frac{-1 \pm 
i \sqrt{3}}{2} = \pm e^{i\frac{2\pi}{3}} \]
\[ \textit{ce sont les racines cubiques de l'unité : } j \textit{ et } j^2 \]

\[ u_n=(an+b)+cj^n+dj^{2n} \]

Calcul de a, b, c et d à partir des conditions initiales
\[ \begin{cases} u_0=0=b+c+d \\ 
u_1=1=a+b+cj+dj^2 \\
u_2=2=2a+b+cj^2+dj \\
u_3=3=3a+b+c+d \end{cases} \]

\[ \Ra \begin{cases} a=1 \\ 
b+c+d=0 \\
b+cj+dj^2=0 \\
b+cj^2+dj=0 \end{cases} \]

\[ \Ra \begin{cases} a=1 \\ 
b=-c-d \\
c(j-1)+d(j^2-1)=0 \\
c(j^2-1)+d(j-1)=0 \end{cases} \]

\[ \Ra \begin{cases} a=1 \\ 
b=c=d=0 \end{cases} \]
 
\end{exercice}
 
\begin{exercice}

\[ \begin{cases} u_n=3u_{n-1}^2 \textit{ pour } n \ge 1 \\ u_0=1 \end{cases} \]

Il faut se ramener à une équation de récurrence linéaire qu'on résoud par la méthode des l'équation caractéristique. On passe donc par les logarithmes.

\[ \log u_n =n\log 3+2\log u_{n-1} \]

On change de varible : $v_n=\log u_n$.
\[ \begin{cases} v_n=2v_{n-1}+n\log 3 \\
v_0=\log u_0=0 \end{cases} \]

Le second membre est bien de la forme $b^nP(n)$ avec $b=1$ et $P(n)=n\log 3$ ($d^o P=1$), on utilise donc la méthode de l'équation caractéristique.

\[ (x-2)(x-2)^2=0 \]
\[ v_n=a2^n+bn+c \]

Calcul de b et de c
\[ bn+c=2[b(n-1)+c]+n\log 3 \]
\[ -bn-c-2b=n\log 3\]
\[ \begin{cases} b=-\log 3 \\ c+2b=0 \Ra c=-2b=2\log 3 \end{cases} \]

Calcul de a à partir de $v_0$
\[ v_0=0=a+c \Ra a=-2\log 3 \]
\[ v_n=-2^{n+1}\log 3-n\log 3+2\log3 \]
\[ v_n=[-2^{n+1}-n+2]\log 3 \]
\[ \Ra u_n=e^{v_n}=e^{[-2^{n+1}-n+2]\log 3} \]
\[ u_n=3^{-2^{n+1}-n+2} \]

\end{exercice}

\begin{exercice}

\[ \begin{cases} T(n)=aT(\frac{n}{n})+g(n) \\ T(1)=1 \end{cases} \]

On obtient ce genre d'équations avec des algos du type diviser pour régner. Problème décomposé en sous problèmes de taille $\frac{n}{b}$. $g(n)$ est le cout de la décomposition, et de la recomposition.

Hypothèse : $n$ puissance de $b$ : $n=b$
\[ \begin{cases} T(b^k)=aT(b^{k-1})+g(b^k) \\ T(b^0)=1 \end{cases} \]

On fait ler changement de variable : $u_k=T(b^k)$
\[ \begin{cases}u_k=au_{k-1}+g(b^k) \\ u_0=1 \end{cases}\]

Par la méthode des facteurs sommants

\[ \begin{cases}
	u_k=au_{k-1}+g(b^k)\\
	u_{k-1}=au_{k-2}+g(b^{k-1})\\
	...\\
	u_k=au_1+g(b)
\end{cases}\]

\[u_k=a^ku_0+\sum_{i=0}^{k-1} a^ig(b^{k-1})\]

\[\Ra u_k=a^k+\sum_{i=0}^{k-1} a^ig(b^{k-1})\]
\end{exercice}

\section{TD2}
\subsection{Series génératrices pour résoudre des équations de récurrence}
\begin{exercice}
\[ \begin{cases}
	u_n = 5u_{n-1} - 6u_{n-2}$ pour n $\ge2\\
	u_0=0,u_1 = 1
\end{cases}\]

Faire apparaitre la génératrice $u(x) = \sum_{n\ge0} u_n x^n$\\

$\ra$ \underline{\textbf{Multiplier par :}} $x^n$\\
$u_nx^n = 5u_{n-1}x^n - 6u_{n-2}x^n$ pour $n \ge 2$\\

$\ra$ \textbf{\underline{Sommer ces équations pour faire apparaitre la série génératrice :}}\\
$\sum_{n\ge2}u_nx^n = 5u_{n-1}x^{n-1} - 6x^2 \sum_{n\ge2} u_{n-2}x^{n-2}$\\

%$u(x)-u_0-u_2x = 5x \sum_{n\ge2}u_{n-1}$\\
%... voir Cannibal : fait
\[ u(x) = \sum_{n>=2}^{} u_nx^n\]
On obtient après sommage
\[ \sum_{n>=2}^{} u_nx^n = 5\sum_{n>=2}^{} u_{n-1}x^{n-1} - 6\sum_{n>=2}^{} u_{n-2}x^{n-2}\]
\[ u(x)-u_0-u_1x = 5x\sum_{n>=2}^{} u_{n-1}x^{n-1} - 6x^2\sum_{n>=2}^{} u_{n-2}x^{n-2}\]
\[ = 5x\sum_{n>=1}^{} u_{n}x^{n} - 6x^2\sum_{n>=0}^{} u_{n}x^{n}\]
\[ u(x)-u_0-u_1x = 5x[u(x)-u_0]-6x^2u(x)\]
$\ra$ avec $u_0 = 0, u_1 = 1$
\[\Ra u(x) - x = 5xu(x) - 6x^2u(x)\]
\[u(x)[6x^2-5x+1]=x\]
\[u(x)=\frac{x}{6x^2-5x+1}\]
$\ra$ racines de $6x^2-5x+1$
\[\delta = 25-24 = 1 >0 \ra 2\] racines distinctes $\frac{5 \pm 1}{12}$ donne $\frac{1}{2}$ ou $\frac{1}{3}$
\[u(x)=\frac{x}{6(x-\frac{1}{2})(x-\frac{1}{3})}=\frac{a}{x-\frac{1}{2}}+\frac{b}{x-\frac{1}{3}}\]
%fini

$\ra X(x-\frac{1}{2})$ puis $x = \frac{1}{2}$ => $a=\frac{\frac{1}{2}}{6(\frac{1}{2}-\frac{1}{3})} =  \frac{1}{2})$\\

$\ra X(x-\frac{1}{3})$ puis $x = \frac{1}{3}$ => $b=\frac{\frac{1}{3}}{6(\frac{1}{3}-\frac{1}{2})} = \frac{1}{3})$\\

$u(x) = \frac{1}{2(x-\frac{1}{2})} - \frac{1}{3(x-\frac{1}{3})} = \frac{1}{2x-1}-\frac{1}{3x-1} = \frac{1}{1-3x} - \frac{1}{1-2x}$\\

$\Ra$ utiliser $\frac{1}{1-cz} = \sum_{n \ge 0}c^nz^n \Ra$ \[ \begin{cases}
	\frac{1}{1-3x} = \sum_{n\ge0}3^nx^n\\
	\frac{1}{1-2x} = \sum_{n\ge0}2^nx^n
\end{cases}\]\\

% Voir Thibaud 2 : fait
\[\Ra u(x) = \sum_{n>=0}^{} u_{n}x^{n}=\sum_{n>=0}^{} 3^{n}x^{n}-\sum_{n>=0}^{} 2^{n}x^{n}\]
\[\sum_{n>=0}^{} (3^{n}-2^{n})x^{n}\]
$\Ra u_{n} = 3^{n}-2^{n}$ pour $n>=0$

\end{exercice}

\begin{exercice}
Même travail sur le système suivant :
\[\begin{cases}
	u_n = 2u_{n-1} + u_{n-2} - 2u_{n-3} pour n>=3\\
	u_0 = 0, u_1 = u_2 = 1\\
\end{cases}\]

\[\sum_{n>=3}^{}u_nx^n = 2\sum_{n>=3}^{}u_{n-1}x^n + \sum_{n>=3}^{}u_{n-2}x^n - 2\sum_{n>=3}^{}u_{n-3}x^n\]
\[u(x)-u_0-u_{1}x-u_{2}x^2 = 2x\sum_{n>=2}^{}u_nx^n+x^2\sum_{n>=2}^{}u_nx^n-2x^3\sum_{n>=2}^{}u_nx^n\]


%fini
$u(x)-u_0 -u_1x - u_2x^2 = 2x [u(x)-u_0-u_2x] + x^2[u(x) - u_0] - 2x^3u(x)$\\
avec $u_0 = 0, u_1 = u_2 = 1$\\

$\Ra u(x) - x - x^2 = 2x u(x) - 2x^2 +x^2u(x) - 2x^3u(x)$\\

$u(x)[2x^3-x^2-2x+1] = x-x^2$\\

$\Ra u(x) = \frac{x(1-X)}{2x^3 - x^2 -2x +1}$\\

racines de $2x^3 -x^2 -2x +1$ :\\
Une racine évidente :\\
$2x^3 - x^2 - 2x + 1 = (x-1)(2x^2+x-1) \\= (x-1)(x+1)(2x-1)$ car -1 racine de $2x^2+x-1$\\



%Voir Thibaud 3 : Fait
\[\Ra u(x) = \frac{x(x-1)}{(x-1)(x+1)(2x-1)}=\frac{x}{(x+1)(2x-1)}=\frac{a}{1+x}+\frac{b}{1-2x}\]
$\ra x(1+x)$ puis $x=-1$ donne $a=\frac{-1}{1+2}=\frac{-1}{3}$

$x(1-2x)$ puis $x=\frac{1}{2}$ donne $b=\frac{\frac{1}{2}}{1+\frac{1}{2}}=\frac{1}{3}$
\[u(x) = \frac{1}{3}\left[\frac{1}{1-2x}-\frac{1}{1+x}\right]\]
utiliser 
\[\frac{1}{1-cz}=\sum_{n>=0}^{} c^{n}z^{n}\]
donne  \[\frac{1}{1-2x}=\sum_{n>=0}^{} 2^{n}x^{n}\]
et \[\frac{1}{1+x}=\sum_{n>=0}^{} (-1)^{n}x^{n}\]
\[u(x) = \frac{1}{3}\left[\sum_{n>=0}^{} 2^{n}x^{n}-\sum_{n>=0}^{} (-1)^{n}x^{n}\right]=\sum_{n>=0}^{} \frac{1}{3}\left[2^n-(-1)^n\right]x^n\]
\[\Ra u_n=\frac{1}{3}\left[2^n-(-1)^n\right]\]pour n $\ge$ 0
\end{exercice}
%fin
\begin{exercice}
\[ \begin{cases}
	u_n=3u_{n-1}-2u_{n-2} $ pour $ n\ge2\\
	u_0=u_1=0
\end{cases}\]

$\sum_{n ge 2}u_nx^n = 3 \sum_{n \ge 2}u_{n-1}x^n - 2\sum_{n \ge 2}u_{n-2}x^n + 4\sum_{n \ge 2}(n-2)x^n$\\
Posons $u(x) =\sum_{n \ge 0} u_nx^n$\\
$u(x)-u_0-u_1x = 3x\sum_{n \ge 2}u_{n-1}x^{n-1} - 2x^2\sum_{n \ge 2}u_{n-2}x^{n-2}+ 4x^2\sum_{n \ge 2}(n-2)x^{n-2}\\
= 3x\sum_{n \ge 1}u_{n}x^n - 2x^2\sum_{n \ge 0}u_nx^n + 4x^2\sum_{n \ge 0}nx^n\\
= 3x[u(x)-u_0] -2x^2u(x)+4x^2\sum_{n \ge 0}nx^n$\\
avec $u_0 = u_1 = 0 $ et $\sum_{n \ge 0}nx^n=\frac{x}{(1-x)^2}$\\

% voir Thibaud 4 : fait
\[u(x)=3xu(x)-2x^2u(x)+\frac{4x^3}{(1-x^2)^2}\]
\[\Ra u(x)[2x^2-3x+1]=\frac{4x^3}{(1-x^2)^2}\]
\[\Ra u(x)=\frac{4x^3}{(1-x^2)^2(2x^2-3x+1)}\]
or \[2x^2-3x+1 = (x-1)(2x-1)\]
\[u(x)=\frac{4x^3}{(1-x^2)^3(2x^2-3x+1)}=\frac{a}{1-2x}+\frac{b}{1-x}+\frac{c}{(1-2x)^2}+\frac{d}{(1-2x)^3}\]
$x(1-2x)$ puis $x=\frac{1}{2}$ donne $a=\frac{4(\frac{1}{2})^3}{(1-\frac{1}{2})^3}=4$

$x(1-x)^3$ puis $x=1$ donne $d=\frac{4}{1-2}=-4$

\[u(x)=\frac{4}{1-2x}-\frac{4}{(1-x)^3}+\frac{b}{1-x}+\frac{c}{(1-x)^2}=\frac{4x^3}{(1-x)^3(1-2x)}\]
%fin

\[\Ra 4x^2 = 4 (1-x)^3 - 4(1-2x)+b(1-2x)(1-x)^2 + c(1-2x(1-x)\\
=4(1-x)(x^2-2x+1) - 4(1-2x) + b(1-x)(2x^2-3x+1) +c(1-3x +2x^2)\\
=4(-x^3 + 3x^2 -3x +1) -4(1-2x) +b(-2x^3 +5x^2 -4x +1) +c(1-3x+2x^2)\\
=x^3[-4-2b] + x^2[12+5b+2c] +x(-4-4b-3c]+b+c\\
b+c=0 \Ra c=-b\\
4+2b = -4 \Ra b=-4 c=4\\
\]
% Thibaud 5 : fait
\[u(x)=4\left[\frac{1}{1-2x}-\frac{1}{1-x}+\frac{1}{(1-x)^2}-\frac{1}{(1-x)^3}\right]\]
\[\frac{1}{1-cx}=\sum_{n>=0}c^nx^n\]
donne \[\frac{1}{1-2x}=\sum_{n>=0}2^nx^n\]
et \[\frac{1}{1-x}=\sum_{n>=0}x^n\]
puis \[\frac{1}{(1-z)^{m+1}}=\sum_{n>=0}\begin{matrix}n+m \\ n\end{matrix} z^n\]

ce qui donne pour m=1 : \[\frac{1}{(1-x)^2}=\sum_{n>=0}\left(\begin{matrix} n+1 \\ n\end{matrix}\right) x^n= \sum_{n>=0}C_{n+1}^nx^n=\sum_{n>=0} (n+1)x^n\]

et pour m=2 : \[\frac{1}{(1-x)^3}=\sum_{n>=0}\left(\begin{matrix} n+2 \\ n \end{matrix}\right) x^n = \sum_{n>=0}C_{n+2}^nx^n=\sum_{n>=0}\frac{(n+1)(n+2)}{2}x^n\]
%fini

\[
\begin{disarray}{r c l}
\Ra u(x) 
&=& \sum_{n \ge 0}u_nx^n\\
&=& 4[\sum_{n \ge 0}2^nx^n -\sum_{n \ge 0}(n+1)x^n - \sum_{n \ge 0}\frac{(n+1)(n+2)}{2}]\\
&=& \sum_{n \ge 0}(2^{n+2} - 4 + 4(n+1 -2(n+1)(n+2))x^n\\
&=& \sum_{n \ge 0}(2^{n+2} -4 -2n(n+1)\\
\end{disarray}\]

$\Ra u_n = 2^{n+2} -2n^2 -2n -4$ pour $n \ge 0$

\end{exercice}

\section{TD3}

\begin{exercice}{Complexité en moyenne du tri rapide en nombre de comparaisons}

Complexité en moyenne du tri rapide en nombre de comparaisons.

$C_n$ : nombre de comparaisons pour trier $n$ éléments
\[ \begin{cases} C_n=(n+1)+\frac{1}{n}\sum_{j=1}^n(C_{j-1}+C_{n-j}) \textit{ pour n } \geq 1 \\ C_0=0 \end{cases}\]

\[ C_n=(n+1)+\frac{1}{n}\left(\sum_{j=1}^nC_{j-1}+\sum_{j=1}^nC_{n-j}\right) \]

\[ C_n=(n+1)+\frac{2}{n}\sum_{k=0}^nC_k\textit{ pour } n \geq 1 \]

$\rightarrow$ Faire apparaître la série génératrice des $C_n$ : $\displaystyle C(x)=\sum_{n\geq0}c_nx^n$

$\ra$ Multiplier par $nx^n$
\[ nC_nx^n = n(n+1)x^n+2(\sum_{k=0}^{n-1}C_k)x^n\]
$\ra$ En sommant
\[ \sum_{n\geq1}nC_nx^n = \sum_{n\geq}n(n+1)x^n + 2(\sum_{k=0}^{n-1}C_k)x^n\]
$\ra$ Différentiation (dérivée)
\[\begin{disarray}{r c l}
\sum_{n\geq1}nC_nx^n
&=& x\sum_{n\geq1}nC_nx^{n-1} \\
&=& x\sum_{n\geq0}(n+1)C_{n-1}x^n \\
&=& xC'(x)
\end{disarray}\]
$\ra$ Ce qui donne au final
\[\begin{disarray}{r c l}
\sum_{n\geq1}^{}n(n+1)x^n
&=&\x\sum_{n\geq1}n(n+1)x^{n-1}\\
&=&x\sum_{n\geq0}n(n+1)(n+2)x^n\\
&=&2x\sum_{n\geq0}\frac{(n+1)(n+2)}{2}x^n\\
&=&\frac{2x}{(1-x)^3}
\end{disarray}\]

\[ \sum_{n\geq 1} \left( \sum_{k=0}^{n-1}C_k\right)x^n
= x\sum_{n\geq 1} \left( \sum_{k=0}^{n-1}C_k\right)x^{n-1}
= x\sum_{n\geq 0} \left( \sum_{k=0}^{n}C_k\right)x^{n}
=x \frac{C(x)}{1-x} \textit{ (somme partielle)} \]

\[ \Ra xC'(x)=\frac{2x}{(1-3x)^3}+\frac{xC(x)}{1-x} \]
\[ \Ra \boxed{ C'(x)=\frac{2}{(1-3x)^3}+2\frac{C(x)}{1-x} } \]
\begin{center}(équation différentielle d'ordre 1)\end{center}

Équation sans second membre :
\[ C'(x)=2\frac{C(x)}{1-x} \Lra \frac{C'(x)}{C(x)}=\frac{2}{1-x} \]
\[ \log \frac{C(x)}{\lambda}=-2\log |1-x|=\log\frac{1}{|1-x|^2} \Ra \boxed{C(x)=\frac{\lambda}{(1-x)^2} } \]

Équation générale $\ra$ méthode de variation de la constante :
\[\begin{disarray}{r c l}
C'(x)
&=&2\frac{\lambda(x)}{(1-x)^2} \Ra C'(x)\\
&=&\frac{\lambda''(x)(1-x)^2+2\lambda(x)(1-x)}{(1-x)^4}\\
&=&\frac{\lambda''(x)}{(1-x)^2}+\frac{2\lambda(x)}{(1-x)^3} 
\end{disarray}\]


\[\Ra\frac{\lambda'(x)}{(1-x)^2}+\frac{2\lambda(x)}{(1-x)^3}=\frac{2}{(1-x)^3}+2\frac{\lambda(x)}{(1-x)^3}\]
\[\Ra\lambda'(x)=\frac{2}{1-x}\]
\[\Ra\lambda(x)=log\frac{1}{(1-x)^2}+K\]

\underline{Calcul à partir de C(0)}
\[\begin{disarray}{r c l}
C(0)=C_0=0 &\Ra& \lambda(0)=0=K+log1=K\\
&\Ra& \boxed{K=0}\\
&\Ra& \lambda(x)=log\frac{1}{(1-x)^2}
\end{disarray}\]

\[\boxed{C(x)=\frac{1}{(1-x)^2}log\frac{1}{(1-x)^2}}\Ra C(x)=\frac{2}{(1-x)^2}log\frac{1}{1-x}\]

\[\begin{disarray}{r c l}
C(x)
&=&\frac{2}{x}\left[\frac{x}{(1-x)^2log\frac{1}{1-x}}\right]\\
&=&\frac{2}{x}n(H_n-1)x^n\\
&=&2\sum_{n\geq0}(n+1)(H_{n+1}-1)
\end{disarray}\]

En utilisant le décalage vers la gauche avec $a_n = n(H_n-1)$ et avec $a_0=0$

\end{exercice}

\begin{exercice}{Calculer le nombre $b_n$ d'arbres binaires ayant $n+1$ feuilles}

$(n+1)$ feuilles $\Lra n$ n\oe ud internes $\Lra (2n+1)$ n\oe uds

$b_0=1 \ra \cdot$

$b_1=1 \ra$ 2 feuilles % + schéma

$b_2=2 \ra$ 3 feuilles % + schéma

Pour $n\geq 1$, on à deux sous arbres : % + schéma
\begin{itemize}
\item un arbre binaire gauche avec k feuilles (avec $1 \leq k \leq n$)
\item un arbre binaire droit avec $(n+1-k)$ feuilles
\end{itemize}

\[ \boxed{ b_n=\sum_{k=1}^n b_{k-1}b_{n-k} } \textit{ pour } n \geq 1 \]

$\ra$ faire apparaître la série génératrice des $(b_n)_{n>0}$

\[b(x)=\sum_{n\geq0}b_nx^n\]

\underline{Multiplier par $x^n$ puis sommer}

\[\sum_{n\geq1}b_nx^n=\sum_{n\geq1}\left(\sum_{k=1}^{n}b_{k-1}b_{n-k}x^n\right)\]
\[
\begin{disarray}{r c l}
b(x)-b(0)
&=&\sum_{n\geq1}\left(\sum_{k\geq0}^{n-1}b_{k-1}b_{n-k}\right)x^n\\
&=&x\sum_{n\geq1}\left(\sum_{k\geq0}^{n-1}b_{k}b_{n-1-k}\right)x^{n-1}\\
&=&x\sum_{n\geq0}\left(\sum_{k\geq0}^{n-1}b_{k}b_{n-k}\right)x^{n}\\
&=&xb(x)b(x)
\end{disarray}
\]
\[\Ra b(x-1)=x[b(x)]^2\Lra \boxed{x[b(x)]^2 - b(x) +1 = 0}\]

$\ra$ équation quadratique

\[ \Delta=1-4x \ra b(x)=\frac{1\pm \sqrt(1-4x)}{2x} \Ra xb(x)=\frac{1\pm \sqrt(1-4x)}{2} \]

Pour que cette égalité soit vrai pour $x=0$, il faut prendre $b(x)=\frac{1\pm \sqrt(1-4x)}{2}$ 
$\ra$ utiliser le développement en série entière de $(1+x)^\alpha$.

\[ (1+x)^\alpha=1+\alpha x+\frac{\alpha(\alpha-1)}{2}x^2 + \dots +\frac{\alpha(\alpha-1)\dots (\alpha-n+1)}{n!}x^n+\dots \]

\[ \Ra \boxed{ (1-4x)^\frac{1}{2}=1-2\sum_{n\geq}\frac{(2n-2)!}{n!(n-1)!}x^n } \]

\[ \Ra b(x)=\frac{1}{2}-\left(\frac{1}{2}-\sum_{n\geq 1}\frac{(2n-2)!}{n!(n-1)!}x^n \right)
\Ra \boxed{ b(x)=\sum_{n\geq 1}\frac{(2n-2)!}{n!(n-1)!}x^n }\]

\[
\begin{disarray}{r c l}
b(x) 
&=& x\sum_{n\geq1}\frac{(2n-2)!}{n!(n-1)!}x^{n-1}\\
&=& x\sum_{n\geq0}\frac{(2n)!}{(n+1)!(n)!}x^n\\
&=& x\sum_{n\geq0}\frac{1}{n+1}\frac{(2n!}{(n)!(n)!}x^n\\
&=& x\sum_{n\geq0}\frac{1}{n+1}\left(\begin{tabular}{l}2n\\n\end{tabular}\right)x^n
\end{disarray}
\]
\[\Ra \boxed{b_n = \frac{1}{n+1}\left(\begin{tabular}{l}2n\\n\end{tabular}\right)}\]

pour $n\geq0 \Ra$ nombre de Catalan

\end{exercice}

\end{document}
