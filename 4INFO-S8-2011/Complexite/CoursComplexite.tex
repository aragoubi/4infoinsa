\providecommand{\VarRectoVerso}{\oneside}
\documentclass[french,11pt]{article}

\usepackage[T1]{fontenc}
\usepackage[utf8]{inputenc}
\usepackage{babel}
\usepackage{fullpage}
\usepackage{lmodern}
\usepackage{graphicx}
\usepackage{enumerate}
\usepackage{xspace}
\usepackage{amsmath}
\usepackage{amsfonts}
%\usepackage{mathenv}
\usepackage{amsthm}
\usepackage[pdfborder={0 0 0}]{hyperref}

\newtheorem{exercice}{Exercice}[section]

\newtheorem{remarque}{Remarque}[subsubsection]
\newtheorem{definition}{Définition}[subsubsection]
\newtheorem{proposition}{Proposition}[subsubsection]
\newtheorem{theoreme}{Théorème}[subsubsection]
\newtheorem{exemple}{Exemple}[subsubsection]

%\usepackage{verbatim}
%\usepackage{moreverb} %pour utiliser plus de fonctions de verbatim

\newcommand{\chapterEt}[1]{\chapter*{#1}\addcontentsline{toc}{chapter}{#1}}
\newcommand{\sectionEt}[1]{\section*{#1}\addcontentsline{toc}{section}{#1}}

\newcommand{\R}{\mathbb{R}}
\newcommand{\x}{\bar{x}}
\newcommand{\ra}{\rightarrow}
\newcommand{\Ra}{\Rightarrow}
\newcommand{\Lra}{\Leftrightarrow}



\title{Complexité}
\author{Danielle Quichaud}
\begin{document}
\maketitle
\tableofcontents
\section{Les séries génératrices pour résoudre les équations de récurrence}
\subsection{Séries génératrices}
Elles permettent de réprésenter une suite infinie de nombres par une fonction. Si de plus, la suite est définie par une équation de récurrence, celle-ci va alors se traduire par un équation fonctionnelle.

\definition{Soit la suite de nombres $(u_n)_{n>=0}$, la série génératrice de $(u_n)_{n>=0}$ est $A(x)=\sum_{n>=0}^{} u_n x^n)$}

x est une variable quelconque, dont les puissances permettent de distinguer les éléments de la suite.
On définit un certain nombre d'opérations sur ces séries indépendamment de leur convergence.

\remarque{Pour les fonctions admettant un dévellopement en séries entière, ce développement est la série génératrice d'une suite.}

\exemple
Developpement en série entière de $e^x$
\\$e^x = 1 + \frac{x}{1!} + \frac{x^2}{2!} + ... + + \frac{x^n}{n!}$
\\
C\'est la série génératrice de la suite $(u_n)_{n>=0} où u_n = \frac{1}{n!} pour n>=0$

\subsubsection{Somme de 2 séries}

Soit $A(x)=\sum_{n>=0}^{} a_n x^n$ la série génératrice de $(a_n)_{n>=0}$\\
Soit $B(x)=\sum_{n>=0}^{} b_n x^n$ la série génératrice de $(b_n)_{n>=0}$

La somme de $A(x) + B(x) = \sum_{n>=0}^{} (a_n + b_n)x^n$

C'est la série génératrice de la suite de $(c_n)_{n>=0}$ avec $c_n = a_n + b_n$

\subsubsection{Produit de cauchy ou produit des convolution}

$A(x)B(x) = \sum_{n>=0}^{} (\sum_{i=0}^{n} a_i b_{n-i})x^n$

C'est la série génératrice de la suite $(c_n)_{n>=0}$ où $c_n = \sum_{i=0}^na_ib_{n-i}$

$a_0 b_0,a_0 b_1 + a_1 b_0, a_0 b_2 + a_1 b_1 + a_2 b_0$

%ce sont des crochets double barre
Si on note $C[x]$ l'ensemble des séries génératrices à coefficient dans C

$C[x]$ muni de ces 2 opérations, somme et produit de Cauchy est un anneau intègre.
\\ \\
\underline{Cas particulier}
%partie 1
$A(x).B(x) = A(x)x^k= \sum_{n>=k}^{} a_{n-k}x^n$

Série génératrice de la suite $(c_n)_{n>=0}$ où
\[ \begin{cases}c_n=0 pour n<k \\
c_n = a_{n-k} pour n>=k \end{cases}\]

$a_0,a_1,...,a_k,...$

$0,0,...    ,a_0,a_1,...$

Décalage de la suite vers la droite

1) $B(x) = x^k$ série génératrice de $(b_n)_{n>=0} où b_k = 1$ et $b_i =0 $ pour  i = k
 
2) $B(x) = 1+ x \Ra $ Série génératrice de $(b_n)_{n>=0}$ avec
\[ \begin{cases}
b_0 = b_1 = 1\\
b_n = 0 pour n >=2
\end{cases}
\\
A(x) ( 1+x) = a + \sum_{n>=1}(a_n + a_{n-1})x^n
\]
C'est la série génératrice de la suite 
\[(c_n)_{n>=0} avec \begin{cases}c_0 = a_0\\c_n = a_n + a_{n-1} pour n >=1\end{cases}\]

%partie 3
$B(x) = 1-x$

Série génératrice de la suite $(b_n)_{n>=0}$ avec 
\[
\begin{cases}
b_0 = 1\\
b_1 = -1\\
b_n = 0 pour n>=2
\end{cases}
\]

$A(x)(1-x) = a_0 + \sum_{n>=1}^{}(a_n - a_{n-1})x^n$

si on prend $(a_n)_{n>=0}$ telle que $a_n=1$ pour $n>=0$ 

alors $A(x) = \sum_{n>=1}^{}x^n$

 et donc $A(x)(1-x)= 1$
 
\subsubsection{Somme partielle}
$\frac{A(x)}{1-x} = (\sum_{n>=0} x^n)(\sum_{n>=0} a_n x^n)$

$\frac{A(x)}{1-x} = \sum_{n>=0} (\sum_{i=0}^n a_{n-i}) x^n$
 
\subsection{Dérivation et intégration}
$\ra$ la série dérivée est obtenue en dérivant terme à terme.

$A(x) = \sum_{n>=0} a_n x^n$ série génératrice de $(a_n)_{n>=0}$

$= a_0 + a_1x + a_2x^2 + a_3x^3 + ...$

$\ra a_1 + 2a_2x + 3a_3x^2 + ....$

$\Ra A'(x) = \sum_{n>=0}(n+1)a_{n+1}x^n$

C'est la série génératrice de la suite \[(c_n)_{n>=0}\] 

où \[c_n = (n+1) a_{n+1}\]

L'intégration se fait de la même façon, terme à terme

\[a_0x + \frac{a}{2}x^2 + \frac{a_2}{3}x^3 +... \]

$\int_{0}^x A(t)dt = \sum_{n>=1}^{}\frac{a_{n-1}}{n}x^n$

C'est la série génératrice de la suite $(c_n)_{n>=0}$ où
\[
\begin{cases}
c_0 = 0\\
c_n = \frac{a_{n-1}}{n} pour n>= 1
\end{cases}
\]

\remarque{Si les séries ont un rayon de convergence non nulle, on démontre que ces opérations coincident avec celles de m nom sur les fonctions.}

\exercice
1) Montrer que
\[ \frac{1}{(1-x)^2} = \sum_{n>=0}(n+1)x^n\]
%resolution de Q
2)
$ log \frac{1}{1-x} = \sum_{n>=1} \frac{x^n}{n}$

$(log \frac{1}{1-x}) est la primitive de \frac{1}{1-x} = \sum_{n>=0} x^n$

$\Ra log \frac{1}{A-x} = \int_O^x (\sum_{n>=0}t^n)dt$

$= \sum_{n>=1}\frac{x^n}{n}$

3) $\frac{1}{1-x}log \frac{1}{1-x} = \sum_{n>=0}H_nx^n$

en effet

$\frac{1}{1-x}log \frac{1}{1-x} = (\sum_{n>=0}x^n)(\sum_{n>=1}\frac{x^n}{n})
= \sum_{n>=0}(\sum_{i>=1}^n \frac{A}{i})x^n = \sum_{n>=0}H_nx^n$

$\ra \frac{1}{(1-x)^2} = \frac{1}{1-x} * \frac{1}{1-x} = (\sum_{n>=0}^{}x^n)(\sum_{n>=0}^{}x^n) = \sum_{n>=0}^{}(\sum_{i=0}^{}1)x^n = \sum_{n>=0}^{}(n+1)x^n$


en utilisant la dérivée

$\ra \frac{1}{(1-x)^2}$ est la dérivée de $\frac{1}{1-x} = \sum_{n>=0}^{}(n+1)x^n$\\

$\frac{1}{(1-x)^2} = (\sum_{n>=0}^{}x^n)' = \sum_{n>=0}^{}(n+1)x^n$

%VOIR exercices GEP
\section{II - Utilisation des séries génératrices pour la résolution des équations de récurrence}

\exemple
\[\begin{cases}
u_n = u_{n-1} +1\\
u_0 = 0
\end{cases}\]
\underline{Multiplier équation par $x^n$}
$u_nx^n = u_{n-1}x^n + x^n $ pour n>=1

\underline{Sommer ces équations} pour faire apparaître la série génératrice des $(u_n)_{n>=0}$

Cette série génératrice est $u(x) = \sum_{n>=0}^{}u_nx^n$
$\sum_{n>=1}^{}u_nx^n = \sum_{n>=1}^{}u_nx^n  + \sum_{n>=1}^{}x^n$

%Indenter les suites d'équation
$u(x) - u_0 = x \sum_{n>=1}^{}u_{n-1}x^{n-1} + \sum_{n>=1}^{}x^{n}$

$= x \sum_{n>=0}^{}u_{n}x^{n} + \sum_{n>=1}^{}x^{n}$

$= xu(x) + \sum_{n>=1}^{}x^{n}$

$= xu(x) + \sum_{n>=0}^{}x^{n}-1$

\exemple{Suite de Fibonacci}
%FAIRE L'INDENTATION
\[\begin{cases}
u_n = u_{n-1} + u_{n-2}\\
u_0 = 0, u_1 = 1
\end{cases}\]

\underline{Multiplier équation par $x^n$}
$u_nx^n = u_{n-1}x^n + u_{n-2}x^n$ pour n>=2

\underline{Sommer}
$\sum_{n>=2}u_nx^n = \sum_{n>=2}u_{n-1}x^n + \sum_{n>=2}u_{n-2}x^n$

\underline{Faire apparître la série génératrice}
$u(x) - u_0 - u_1x = x \sum_{n>=2}u_{n-1}x^{n-1} + x^2 \sum_{n>=2}u_{n-2}x^{n-2}$

$= x \sum_{n>=1}u_{n}x^{n} + x^2\sum_{n>=0}u_{n}x^{n}$

$= x (u(x) - u_0) + x^2u(x)$

$(x^2 + x -1)u(x) = -u_0 -u_1x + u_0x$

$u(x) = \frac{(u_0 - u_1)x-u_0}{x^2+x-1}$

%voir GEP pour la fin de cet exemple

%fin après GEP
$a = \frac{-r_1}{r_1-r_2} = \frac{\frac{1-\sqrt{5}}{2}}{\sqrt{5}}$

multiplier par $(x - r_2)$ puis faire $x=r_2$

$b = \frac{-r_2}{r_2-r_1} = \frac{r_2}{r_1-r_2} = \frac{\frac{-1-\sqrt{5}}{2}}{\sqrt{5}} = \frac{-1-\sqrt{5}}{2\sqrt{5}}$

$u(x) = \frac{a}{x-r_1} + \frac{b}{x-r_2} = \frac{-a}{r_1(1-\frac{x}{r_1})} - \frac{b}{r_2(1-\frac{x}{r_2})}$

en utilisant $\frac{1}{1-cz} = \sum_{n>=0}c^nz^n$

on obtient $u(x) = \frac{-a}{r_1} \sum_{n>=0}(\frac{1}{r_1})^n x^n - \frac{b}{r2} \sum_{n>=0}(\frac{1}{r_2})^n x^n$\\
on a $r_1r_2 = -1 <=> r_1 = -\frac{1}{r_2} et r_2 = -\frac{1}{r_1}$\\
$\Ra u(x) = a r_2 \sum_{n>=0}(-r_2)^nx^n + br_1 \sum_{n>=0}(-r_A)^nx^n
\Ra u(x) = \sum_{n>=0}u_nx^n = \sum_{n>=0}[ar_2(-r_2)^n + br_1(-r_1)^n] x^n$

$ar_2 = -\frac{r_1r_2}{\sqrt{5}} = \frac{1}{\sqrt{5}}$

$br_1 = \frac{r_1r_2}{\sqrt{5}} = -\frac{1}{\sqrt{5}}$

$u_n = \frac{1}{\sqrt{5}}((\frac{1+\sqrt{5}}{2})^n-(\frac{1-\sqrt{5}}{2})^n)$ pour n>=0

\end{document}




