
\providecommand{\VarRectoVerso}{\oneside}
\documentclass[french,11pt]{article}

\usepackage[T1]{fontenc}
\usepackage[utf8]{inputenc}
\usepackage{babel}
\usepackage{fullpage}
\usepackage{lmodern}
\usepackage{graphicx}
\usepackage{enumerate}
\usepackage{xspace}
\usepackage{amsmath}
\usepackage{amsfonts}
%\usepackage{mathenv}
\usepackage{amsthm}
\usepackage[pdfborder={0 0 0}]{hyperref}

\newtheorem{exercice}{Exercice}[section]

\newtheorem{remarque}{Remarque}[subsubsection]
\newtheorem{definition}{Définition}[subsubsection]
\newtheorem{proposition}{Proposition}[subsubsection]
\newtheorem{theoreme}{Théorème}[subsubsection]
\newtheorem{exemple}{Exemple}[subsubsection]

%\usepackage{verbatim}
%\usepackage{moreverb} %pour utiliser plus de fonctions de verbatim

\newcommand{\chapterEt}[1]{\chapter*{#1}\addcontentsline{toc}{chapter}{#1}}
\newcommand{\sectionEt}[1]{\section*{#1}\addcontentsline{toc}{section}{#1}}

\newcommand{\R}{\mathbb{R}}
\newcommand{\x}{\bar{x}}
\newcommand{\ra}{\rightarrow}
\newcommand{\Ra}{\Rightarrow}
\newcommand{\Lra}{\Leftrightarrow}



\title{Complexité}
\author{Danielle Quichaud}
\usepackage{amsmath}
\begin{document}
\maketitle
\tableofcontents


\section{Les séries génértrices pour résoudre les équations de récurrence.}
\subsection{Séries génératrices}
Elles permettrent de représenter une suite infime par une fonction si de plus la suite est définie par équation de récurence, celle ci, celle-ci va alors se traduire par une équation fonctionnelle.

\definition
Soit la suite de nombres $(u_n)_n>=0$, la série génératrice de $(u_n)_n>=0$ est $A(x) = \sum_{n>=0}u_nx^n$

Remarque : Pour les fonctions admettant un dévellopement en séries entière, ce développement est la série génératrice d'une suite.

\exemple
Developpement en série entière de $e^x$
\\$e^x = 1 + \frac{x}{1!} + \frac{x^2}{2!} + ... + + \frac{x^n}{n!}$
\\
C'est la série génératrice de la suite $(u_n)_{n>=0} où u_n = \frac{1}{n!} pour n>=0$
\subsubsection{A) Somme de 2 series}


\subsubsection{B) Produit de Cauchy ou produit de convolution}

$A(x)B(x) = \sum_{n>=0}(\sum_{i=0}^{n}a_i b_{n-i}) x^n$\\
C'est la série génératrice de la suite $(c_n)_{n>=0}$ où $c_n = \sum_{i=0}^na_ib_{n-i}$
%blabla avec ab 

\underline{Cas particulier}
1) $B(x) = x^k$ série génératrice de $(b_n)_{n>=0} où b_k = 1 et b_i =0 pour i \\= k$
 
2) $B(x) = 1+ x \Ra $ Série génératrice de $(b_n)_{n>=0}$ avec
\[ \begin{cases}
b_0 = b_1 = 1\\
b_n = 0 pour n >=2
\end{cases}
\\
A(x) ( 1+x) = a + \sum_{n>=1}(a_n + a_{n-1})x^n
\]
C'est la série génératrice de la suite 
\[(c_n)_{n>=0} avec \begin{cases}c_0 = a_0\\c_n = a_n + a_{n-1} pour n >=1\end{cases}\]
\\
%petit 3, Q
$
\Ra \sum_{n>=0} x^n = frac{1}{1-x}
$
\subsubsection{Somme partielle}
\[
\frac{A(x)}{1-x} = (\sum_{n>=0} x^n)(\sum_{n>=0} a_n x^n)
\\
\frac{A(x)}{1-x} = \sum_{n>=0} (\sum_{i=0}^n a_{n-i}) x^n
\]
\subsection{c) Dérivation et intégration}\\

\[\ra la série dérivée est obtenue en dérivant terme à terme.\\
A(x) = \sum_{n>=0} a_n x^n série génératrice de (a_n)_{n>=0}\\
= a_0 + a_1x + a_2x^2 + a_3x^3 + ...
\\  
\ra a_1 + 2a_2x + 3a_3x^2 + ....\\
\Ra A'(x) = \sum_{n>=0}(n+1)a_{n+1}x^n\\
C'est la série génératrice de la suite (c_n)_{n>=0} où c_n = (n+1) a_{n+1}
\]
%blabla Q sur les intégrales
\exercice
1) Montrer que\\
\[ \frac{1}{(1-x)^2 = \sum_{n>=0}(n+1)x^n\]
%resolution de Q
2)
\[ log \frac{1}{1-x} = \sum_{n>=1} \frac{x^n}{n}\\ \]
(log \frac{1}{1-x}) est la primitive de \frac{1}{1-x} = \sum_{n>=0} x^n

\Ra log \frac{1}{A-x} = \int_O^x (\sum_{n>=0}t^n)dt
= \sum_{n>=1}\frac{x^n}{n}

3) \frac{1}{1-x}log \frac{1}{1-x} = \sum_{n>=0}H_nx^n\\
en effet
\frac{1}{1-x}log \frac{1}{1-x} = (\sum_{n>=0}x^n)(\sum_{n>=1}\frac{x^n}{n})\\
= \sum_{n>=0}(\sum_{i>=1}^n \frac{A}{i})x^n = \sum_{n>=0}H_nx^n\\
%Partie 2, Q
\section{II) Utilisation des séries génératrices pour la résolution d'équations de récurrences}

or $u_0$ = 0\\

$(1-x)u(x) = \frac{1}{1-x} - 1$\\
$=\frac{x}{1-x}^2$

or $\frac{x}{(1_x)^2} = \sum_{n>=1} n x^n\\
donc on a trouvé que u(x) = \sum_{n>=0} u_n x^n = \sum_{n>=1} n x^n

\begin{cases}
u_0 = 0\\ u_n = n pour n>=1
\end{cases} 
%Exemple 2 : Suite de fibonnacci : Q
[\
u(x) = \frac{-x}{x^2+x-1}
\]
Fraction de polynomes que l'on va décomposer en éléments simples
$x^2+x-1 = 0\\
\delta = 1 +4 = 5>0 \ra 2 racines distinctes
r_1 = \frac{-1 + \sqrt{5}}{2} et r_2 = \frac{-1 - \sqrt{5}}{2}\\

u(x) = \frac{a}{x-r_1} + \frac{b}{x-r_2} = \frac{-x}{(x-r_1)(x-r_2)}\\
\ra \underline{Calcul de a et b}\\
* a(x-r_2) + b(x-r_1) = -x
* multiplier par (x - r_1) pour faire x = r_1

%tableau de Q

on obtient $u(x) = \frac{-a}{r_1} \sum_{n>=0}(\frac{1}{r_1})^n x^n - \frac{b}{r2} \sum_{n>=0}(\frac{1}{r_2})^n x^n$\\
on a $r_1r_2 = -1 <=> r_1 = -\frac{1}{r_2} et r_2 = -\frac{1}{r_1}$\\
$\Ra u(x) = a r_2 \sum_{n>=0}(-r_2)^nx^n + br_1 \sum_{n>=0}(-r_A)^nx^n
\Ra u(x) = \sum_{n>=0}u_nx^n = \sum_{n>=0}[ar_2(-r_2)^n + br_1(-r_1)^n] x^n$
% tableau Q 

\end{document}