\documentclass{article}

\usepackage[utf8]{inputenc}
\usepackage[frenchb]{babel}
\usepackage{listings}
\usepackage{supertabular}

\begin{document}

\title{
		Réseau Laurent Bedat - Correction TD
	}
\author{Guillaume Pannetier}
\date{\today}

\maketitle

%---------------------
%  Classes de réseau
%---------------------
\section{Classes de réseau}

\begin{center}
	\begin{supertabular}{|c|l|}
		\hline
		\textbf{Classe} & \textbf{Caractéristiques} \\
		
		\hline
		Classe A & Seul le $1^{er}$ octet indique le réseau (de 1 à 126, 127 = boucle locale) \\
		
		\hline
		Classe B & Les deux premiers octets pour le réseau. Le premier octet va de 128 à 192 \\
		
		\hline
		Classe C & Les trois premiers octets pour le réseau.\\

		\hline		
	\end{supertabular}
\end{center}


%--------------------
% Routage RIP
%--------------------
\section{Table de routage RIP}
\subsection{\'Evolution pour le routeur 2}
\begin{supertabular}{|rrll|}
	\hline

	\textbf{Temps} & & \textbf{Routes} & \textbf{Co\^ut} \\
	
	\hline

	$T_{0}$ & R & 24.0.0.0 & 0 \\
	& R & 113.0.0.0 & 0 \\
	& R & 114.2.0.0 255.255.0.0 & 0 \\
	
	\hline
	
	$T_{1}$ & +R & 193.57.96.0 gw router1 & 1 \\
	& & 193.57.97.0 gw router1 & 1 \\
	& +R & 114.3.0.0 gw routerCentral & 1 \\
	& & 114.1.0.0 gw routerCentral & 1 \\
	& & 114.250.0.0 gw routerCentral & 1 \\
	
	\hline
	
	$T_{2}$ & +R & 163.44.0.0 gw routerCentral & 2 \\
	
	\hline
	
	$T_{3}$ & & Tout le réseau est connu & \\
	
	\hline
\end{supertabular}

\subsection{Panne du routeur 4}
\paragraph{Détection de la panne :}
\begin{itemize}
	\item Soit le routeur voisin sait qu'il n'y a plus d'informations venant de ce routeur et supprime alors les routes qui y passent.
	\item En réalité, une durée de vie est fixée pour les entrées des tables de routage.
\end{itemize}

\section{Découpage en sous-réseaux}
On souhaite découper le réseau 163.44.0.0 en 16 sous-réseaux.\\
$\underbrace{163.44}.\underbrace{0.0}$ peut adresser 65 000 machines.\\

Il va donc falloir prendre 4 bits pour identifier le sous réseau.

\paragraph{Masque de sous-réseau :} $\underbrace{1111 1111.1111 1111.1111} \underbrace{0000.0000 0000}$ = 255.255.240.0

\paragraph{Remarque :}Le 0 et le 255 étant réservés dans les adresses pour l'adresse réseau et l'adresse broadcast, il faut toujours retirer 2 au nombre de machines adressables.

\end{document}