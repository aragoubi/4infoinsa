\section{Comptabilité générale}
\subsection{Les documents de synthèse}
La comptabilité général répond à une \textbf{obligation légale}. Les documents de synthèses sont \textbf{publics} car déposés au greffe du tribunal de commerce.

\subsubsection{Le bilan}
Le bilan est établi à la fin de l'exercice comptable (en général 12 mois). Il donne \textbf{à une date donnée la situation patrimoniale d'une entreprise}. Il permet de lire \textbf{ce que possède et ce que doit l'entreprise} mais aussi de conna\^itre \textbf{d'où vient le financement et ce que l'entreprise fait de ces ressources}. \textbf{On peut le voir comme une photographie du patrimoine}.\\

\paragraph{Remarques :}
\begin{itemize}
	\item Les actifs et les passifs d'un bilan doivent obligatoirement être équilibrés !
	\item Les actifs sont triés par liquidité (le moins liquide en haut).
\end{itemize}

\paragraph{Quelques définitions :}
\begin{itemize}

	\item \textbf{Le patrimoine} d'une personne morale ou physique est la différence entre ses biens et créances et ses dettes.
	
	\item \textbf{La valeur nette du patrimoine = situation nette} = valeur de l'ensemble des biens (Actifs) - valeur de l'ensemble des dettes envers les tiers.
	
	\item \textbf{Les capitaux propres} sont les ressources propres à l'entreprise qui lui sont affectées d'une manière durable par les associés. Dette fictive puisqu'ils appartiennent aux associés (qui sera remboursée si l'entreprise ferme et que les autres dettes sont remboursées). Ils comprennent le \textbf{capital}, les \textbf{réserves} et le \textbf{résultat de l'exercice}.
	
\end{itemize}

\subsubsection{Le compte de résultat}
Le compte de résultat permet \textbf{d'analyser comment l'entreprise créé de la richesse}. D'un c\^oté se trouvent \textbf{les charges} (~les dépenses) et de l'autre \textbf{les produits} (~les recettes).


\subsection{Tableau d'amortissement}
Il sert à conna\^itre le co\^ut d'un pr\^et (= le montant total des intér\^ets).\\

\textbf{Exemple par amortissement constant :}\\
\begin{tabular}{ll}
	Capital emprunté : & 30 000 euros \\
	Périodicité : & mensuelle \\
	Durée totale : & 6 mois \\
	Taux d'intér\^et annuel : & 8\% \\
	Date de réalisation : & $1^{er}$ janvier N+1
\end{tabular}


\begin{tabular}{|c|c|c|c|c|}

	\hline
	
	ANNEE & Capital restant d\^u en début de mois & Amortissement & Intér\^ets & Mensualité \\
	
	\hline
	
	Janvier & 30 000 & 5000 & 200,00 & 5200,00 \\
		
	\hline
	
	Février & 25 000 & 5000 & 166,67 & 5166,67 \\
		
	\hline
	
	Mars & 20 000 & 5000 & 133,33 & 5133,33 \\
		
	\hline
	
	Avril & 15 000 & 5000 & 100,00 & 5100,00 \\
		
	\hline
	
	Mai & 10 000 & 5000 & 66,67 & 5066,67 \\
		
	\hline

	Juin & 5 000 & 5000 & 33,33 & 5033,33 \\
		
	\hline
	
	TOTAL &	& 30 000 & 700,00 & \\
	
	\hline
	
\end{tabular}


\paragraph{Erreurs à éviter :}
\begin{itemize}
	\item Capital restant d\^u au début du mois M\\
	= (Capital restant M-1) - (mensualité M-1) $\Longleftarrow$ \textbf{FAUX !}\\
	= (capital restant M-1) - (Amortissement M-1) $\Longleftarrow$ \textbf{CORRECT !}
	
	\item Intérêts\\
	= capital emprunté * taux d'intérêt mensuel $\Longleftarrow$ \textbf{FAUX !}\\
	= capital restant M * taux d'intér\^et mensual $\Longleftarrow$ \textbf{CORRECT !}
\end{itemize}

\paragraph{Quelques définitions :}
\begin{itemize}
	\item \textbf{Emprunt obligatoire} : Emprunt auprès de plusieurs pr\^eteurs;
	\item \textbf{Emprunt indivis} : Emprunt auprès d'un seul pr\^eteur;
	\item \textbf{Capital :} somme emprunter et à rembourser;
	\item \textbf{Intérêt} : Coût de l'argent prêtée (et rémunération pour la banque);
	\item \textbf{Amortissement} : Par du capital remboursé chaque année;
	\item \textbf{Mensualité} : Montant total de l'échéance versée à la banque (amortissement + intérêt).
\end{itemize}


\subsection{La Valeur Ajoutée}
\paragraph{Définition :} Différence entre les biens ou services \textbf{produits} par une entreprise et celle des biens et services \textbf{achetés} pour la production, c'est à dire les \textbf{consommations intermédiaires}.\\

\textbf{VA = CAHT - consommations intermédiaires}\\

\textbf{Richesse nette créée} par l'entreprise du fait de son activité.


\subsection{La TVA}
\paragraph{Qui verse la TVA ?} "Personne qui effectue de manière \textbf{habituelle} des activités économiques".

\begin{itemize}
	\item Commerçant / artisan / industriel / profession libérale
	\item \`A l'exception de :
		\begin{itemize}
			\item Quelques opérations de banque / assurance
			\item Activités médicales
			\item Activités d'enseignement
		\end{itemize}
\end{itemize}

\subsubsection{\`A savoir sur la TVA}
\begin{itemize}
	\item \textbf{En terme de résultat}
	\begin{itemize}
		\item La TVA n'est \textbf{jamais un co\^ut pour l'entreprise}
		\item La TVA n'est \textbf{jamais un produit pour l'entreprise}.
		\item $\Rightarrow$ On dit que la TVA \textbf{est neutre}\\
	\end{itemize}
	
	
	\item \textbf{En terme de trésorerie}
	\begin{itemize}
		\item La déclaration de TVA est faite \textbf{"posteriori"}, elle est donc remboursée par l'état le mois suivant $\Longrightarrow$ impact possible sur la trésorerie.\\
	\end{itemize}
	
	\item La TVA est un \textbf{imp\^ot indirect} : Part versée aux entreprise reversée à l'état.
	\item La TVA est un \textbf{impôt sur la consommation} : Seul le consommateur la paie réellement.
\end{itemize}

\subsection{FR, BFR et tout le tralala}
\paragraph{Quelques définitions :}
\begin{itemize}

	\item Un \textbf{bilan fonctionnel} est une image du fonctionnement de l'entreprise conçue par les \textbf{analystes financiers}.
	
	\item \textbf{L'investissement} correspond à \textbf{l'accroissement des immobilisation} d'une année sur l'autre.

	\item Le \textbf{cycle d'exploitation} correspond aux opérations qui découlent d'une activité courante.
	
\end{itemize}
	
\subsubsection{Besoins de financement}
On distingue deux types de besoins de financement :\\
\begin{itemize}
	\item Celui \textbf{lié à l'investissement} :\\
		L'investissement affecte le "haut de bilan" et correspond à une \textbf{opération à long terme}.\\
		$\Rightarrow$ Devra être financé par des \textbf{ressources stables}.\\

	\item Celui \textbf{lié au cycle d'exploitation} :\\
	Ce besoin de financement vient des décalages dans le temps entre les encaissements et les décaissements, plus précisément l'existance de stocks et de créances. Le \textbf{crédit fournisseur} vient \textbf{réduire le besoin de financement}, appelé \textbf{besoin en fonds de roulement (BFR).}\\
		
	\paragraph{Comment réduire le "décalage à financer" ?}
	\begin{itemize}
		\item en augmentant le crédit fournisseur;
		\item en diminuant le crédit client;
		\item en réduisant le niveau des stocks (flux tendus, juste-à-temps, ...)
	\end{itemize}
\end{itemize}


\subsubsection{Le bilan fonctionnel}

\begin{tabular}{c|c|c|c}
	&	Actif	&	Passif	&	\\
	
	\hline
	
	Emplois stables & Actif immobilisé & Capitaux propres	& Ressources stables \\

	\cline{3-3}	
	
	& &	Emprunts & \\
	
	\cline{1-2}
	
	/// &	Stocks	&	& \\
	
	\cline{2-4}
	
	/// &	Créances diverses	&	Dettes diverses	& /// \\
	
	\cline{4-4}
	
	/// &	&	Concours bancaires de trésorerie	&	Trésorerie (ressources) \\
	
	\cline{1-3}
	Trésorerie (emplois) &	VMP	&	& \\
	
	&	Disponibilités	&	& \\
	
	\hline	
	
\end{tabular}\\

\begin{itemize}
	\item \textbf{Fonds de roulement} = Ressources stables - emplois stables
	\item \textbf{Besoin en fonds de roulement} = Stocks et créances diverses - Dettes diverses
	\item \textbf{Trésorerie (solde)} = FR - BFR = Trésorerie (emplois) - Trésorerie (ressources)
\end{itemize}